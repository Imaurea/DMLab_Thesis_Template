% ****************************************************************************************************
%
% DOCUMENT CONFIGURATION AND PACKAGES
%
% ****************************************************************************************************

% LANGUAGE
% ----------------------
\usepackage[english]{babel}
%\usepackage[spanish,es-tabla]{babel}
%\selectlanguage{spanish}

\usepackage[T1]{fontenc} % para colocar acentos y simbolos en español, p.e.: la �
\usepackage[utf8]{inputenc} % para colocar acentos y simbolos en espa�ol, p.e.: la �
%\usepackage[latin1]{inputenc} % Caracteres con acentos.


% SYMBOLS
% ----------------------
\usepackage{latexsym} 
\usepackage{latexsym} 
\usepackage{units} % Units
\usepackage{siunitx} % SI units
\usepackage{amssymb} 
\usepackage{amsfonts}
\usepackage{eurosym} % € symbol
\usepackage{wasysym} %Additional math symbols
\usepackage{esvect} %vectors

\usepackage{tikz} %to write numbers/letters inside a circle. Use \circled{}
\newcommand*\circled[1]{\tikz[baseline=(char.base)]{
            \node[shape=circle,draw,inner sep=2pt] (char) {#1};}}


% LINE SPACING
% ----------------------
\usepackage{setspace} % to adjust interlineate
%\setlength{\parskip}{5mm} %separación de una linea después de un punto y a parte. 5mm equivale a una linea escrita.
\renewcommand{\baselinestretch}{1.5} %interlineado del documento

\def\wl{\par \vspace{\baselineskip}} %when using \wl, space left after the text... similar to \\

% DOCUEMNT MARGINS
% ----------------------
\usepackage[a4paper,top=2.5cm,bottom=3cm,left=2.5cm,right=2cm,marginparsep=0cm,headsep=1cm,footskip=1.7cm]{geometry} 
%showframe to see frame of margins

%\usepackage{anysize} % Soporte para el comando marginsize
%\marginsize{3.5cm}{2.5cm}{2cm}{2cm} %izq, dcha, arriba, abajo

\usepackage{indentfirst} %for indentation of paragraphs
\usepackage{scrextend} % for indentation of a paragraph


% PAGESTYLE HEAD AND FOOT
% ----------------------
% with 2 side documents
% R=Right ; C=Centre ; L=Left 
% O=Odd (Impar) ; E=Even (Par)

\usepackage{fancyhdr} % para dar formato al documento
\pagestyle{fancy}
\fancyhf{}
\fancyhead{} % clear all fields

% Head
%\fancyhead[RO,LE]{\chaptertitlename}
\fancyhead[RO]{\rightmark} %
\fancyhead[LE]{\leftmark} %
\renewcommand{\headrulewidth}{0.4pt}

% Page foot
\fancyfoot[RO,LE]{\thepage}
\fancyfoot[RE]{\titlename}
\renewcommand{\footrulewidth}{0.4pt}


% First page of the chapter
% To apply use: \thispagestyle{plain}
\fancypagestyle{plain}{%
\fancyhf{} % clear all header and footer fields
\fancyfoot[RO,LE]{\thepage}
\renewcommand{\headrulewidth}{0pt}
\renewcommand{\footrulewidth}{0.4pt}}


% TIPOGRAFIA
% ----------------------
%\usepackage{mathpazo} % cod.: ppl / Palatino
%\usepackage{mathptmx} % cod.: ptm / Times
\usepackage{helvet}    % cod.: phv / Helvetica
%\usepackage{avant}     % cod.: pag / Avant Grade
%\usepackage{courier}   % cod.: pcr / Courier
%\usepackage{bookman}  % cod.: pbk / Bookman
%\usepackage{newcent}  % cod.: pnc / New Century
%\usepackage{charter}  % cod.: bch / Charter
%\usepackage{chancery}  % cod.: pzc / Zapf Chancery


% TITLES
% ----------------------
\usepackage{titlesec}
\titleformat{\chapter}[display]
    {\normalfont\huge\bfseries}{\chaptertitlename\ \thechapter}{20pt}{\Huge} %size of title/chapter
\titlespacing*{\chapter}{0pt}{0pt}{45pt} %space between beginning and Chapter X

%\usepackage{titletoc}

%\usepackage{chngcntr} %to make counting chapter changes
%\counterwithin{section}{part}
%\renewcommand{\thepart}{\Roman{part}} %Restart counting at each chapter


% LIST OF CONTENTS
% ----------------------
\usepackage{tocloft}

\renewcommand\cftchapafterpnum{\vskip0.3pt} %espacio en toc
\renewcommand\cftsecafterpnum{\vskip0.5pt}

% BIBLIOGRAPHY
% ----------------------
\usepackage[style=ieee,sorting=none,natbib=false]{biblatex} %bibliography cited with numbers, ordered as appears in the text
%style=numeric,bibencoding=ascii

\addbibresource{references.bib} % The filename of the bibliography


% FIGURES
% ----------------------
\usepackage{graphicx} % Include figures files
\usepackage{graphics}
\usepackage{caption} %for naming tables, figures
%\usepackage{subcaption} %for naming figures inside a figure environment
\usepackage{mwe} 
\usepackage{scalerel} %for scaling inline figures, to have the same height as the text
\usepackage{float} % para que las figuras se coloquen donde queramos
\usepackage{subfig} % use for side-by-side figures
\graphicspath{{Figures/}} % Indica a LateX donde esta la carpeta de las figuras
%\usepackage{chngcntr} %for numbering figures with chapter
%\newcounter{figure}[chapters]
%\counterwithin{figure}{chapters}
%\counterwithin{table}{section}

%\begin{figure}[h]  % "h" es para fijar a figura en la posición definida.
%	\centering
%	\includegraphics[width=0.8\textwidth]{Nombre del archivo}
%	\caption{Título de la figura} % Título debajo de la fotografia
%	\label{Nombre para referenciar la figura}
%\end{figure}


% TABLES
% ----------------------
\usepackage{tabularx}
\usepackage{multirow} %to make text occupy 2 lines in a table
% \multirow{number rows}{width}{text}

% LISTS
% ----------------------
\usepackage{enumerate} % numerar listas
%\begin{itemize}
%	\item 
%\end{itemize}
\usepackage{enumitem} %tochange propoerties of enumerate/itemize


% NOTES IN THE TEXT / MISSING FIGURES WARNINGS
% ----------------------
\usepackage[colorinlistoftodos,textsize=scriptsize,textwidth=1.8cm]{todonotes}
%\listoftodos 
%\todo[options]{todo text}
%\missingfigure{Make a sketch of the structure.}

% EXTRAS
% ----------------------
\usepackage{blindtext} % para agregar texto aleatorio de relleno
%\usepackage{ragged2e} % alinea texto a la izquierda
\usepackage{eqparbox}% to make text occupy the same amount of horizontal space on the page
%\usepackage{blindtext} % para agregar texto aleatorio de relleno

\usepackage{emptypage}

\usepackage{multicol} %multiple columns text

% COLOURS
% ----------------------
\usepackage{color} % paquete para utilizar colores en el documento

%\definecolor{MUcolor}{rgb}{0.02,0.86,1.05}
\definecolor{MUcolor}{cmyk}{0.97,0.30,0.41,0}
%\definecolor{MUcolor}{cmyk}{0.981,0.181,0,0.588}
%\definecolor{naranja}{rgb}{1,0.5,0}
%\definecolor{naranja}{cmyk}{0,0.5,1,0}
%\definecolor{nombre}{gray, rgb o cmyk}{descripción}

%\textcolor[rgb]{1,0,0}{Rojo}\\
%\textcolor[rgb]{1,1,0}{Amarillo}\\
%\textcolor[rgb]{0.2,0.5,0.7}{Azulado}\\
%\textcolor[cmyk]{0,1,0,0}{Magenta}\\
%\textcolor[cmyk]{1,0,1,0}{Verde}\\
%\textcolor[gray]{0.3}{Gris Oscuro}\\
%\textcolor[gray]{0.8}{Gris Claro}\\




% HYPERLINKS
% ----------------------
%\usepackage[colorlinks=true, allcolors=black]{hyperref} %if you want all the links to be black
\usepackage[pdftex=true,colorlinks=true,plainpages=false]{hyperref} 
\usepackage{hyperref} %ALWAYS THE LAST PACKAGE TO UPLOAD!!
\hypersetup{
	hidelinks, % para quitar las cajsa rojas y azules de las referencias
	backref=true,
	pagebackref=true,
	hyperindex=true,
	breaklinks=true,
	colorlinks=true,
	linkcolor=MUcolor,
	citecolor=MUcolor,
	urlcolor=blue,
	bookmarks=true,
	bookmarksopen=false,
	backref=page} % link con la pagina donde cada referencia es citada
	