%----------------------------------------------------------------------------------------
% CHAPTER 01 
%----------------------------------------------------------------------------------------
\setcounter{page}{1}

\chapter{INTRODUCTION} % Main chapter title
\label{Introduction} % for referencing this chapter elsewhere, use \ref{Introduction}

%%\hrule height 0.2pt % linea bajo le titulo

\vspace*{3\baselineskip} %leave space between title an text

\noindent \textit{To introduce introduction text at the beginning of the chapter, to make a description of it. With no indentation.}

%----------------------------------------------------------------------------------------
%	SECTION 1
%----------------------------------------------------------------------------------------
\section{1st section}

La tesis se ha estructurado en siete capítulos, incluyendo este primero como introductorio.

En el \textbf{capítulo 1} se realiza una introducción general al documento del presente trabajo, donde se explica la problemática principal, las motivaciones personales y profesionales de esta tesis, así como la planificación y estructuración de la tesis doctoral. \footnote{This is the example of a footnote.}

El \textbf{capítulo 2} \todo[color=yellow]{Esto sería una nota, en una linea \ldots} recoge la revisión literaria realizada, enfocada en la optimización de las estructuras metálicas, y fundamentalmente la optimización de las uniones semi-rigidas atornilladas y los métodos utilizados para tal fin. \cite{Chander1983AbnormalStudy}

En el \textbf{capítulo 3} se definen los objetivos marcados para dar solución a la problemática planteada, así como las hipótesis formuladas para alcanzar dichos objetivos. 

En el \textbf{capítulo 4} se describe la metodología desarrollada para la optimización de estructuras metálicas.
\blindtext

En el \textbf{capítulo 5} se exponen y analizan los resultados obtenidos.

El \textbf{capítulo 6} recoge las conclusiones y la discusión de los resultados, así como del proyecto en conjunto, y define las aportaciones realizadas.

Finalmente se incluyen las referencias bibliográficas en orden alfabético o de aparición, así como los apéndices con datos e información de apoyo a la memoria.

\section{Figure section}

\blindtext
\blindtext
This is a bibliography cite: \cite{Silvers2010ASplines}, \cite{Chander1983AbnormalStudy}

\blindtext
\blindtext
\subsection{Subsection 2}
\missingfigure{Make a sketch of the structure.}
\blindtext
\begin{figure}[!htb]\centering	
\includegraphics[width=0.69\linewidth]{Figures/Logo_DMLab2.png}
    \caption{Réactions aux roulements (sens 2). FE 186 vs. FE 88}
    \label{reactf_FE88}
\end{figure} 

\blindtext

\begin{table}[h!]
\centering
\captionof{table}{Caractéristiques de la transmission (engrenages) - M7 }
  \begin{tabular}{ c c c }
  \hline\hline\\
  Entraxe & mm & 450,78\\
  Ratio de transmission & & 3,81\\
  Nombre des dents pignon & & 27\\
  Nombre des dents roue & & 103\\
  Module & & 6,7\\
  Coefficient de rayon de pied & & 0,3939 \\[1ex]
  \hline
  Rapport de longueur de dépouille & & 0,822\\
  Dépouille de tête linéaire & $\mu m$ & 67\\
  Bombé & µm & 7\\[1ex]
  \hline\hline
  \end{tabular}    
\label{carac_M7_2}
\end{table}

%% SECTION 01
%% ---------------------------------------------
%\section{NOMBRE DE LA SECCIÓN 1}
%\label{Section01_01}
%% For referencing this section elsewhere, use \ref{SectionX_Y}
%% where X is the number of the chapter,
%% and Y is the number of the section.
%
%Aquí se comienza a escribir el texto de la sección en cuestión.


% SECTION XX
% ---------------------------------------------
%\section{NOMBRE DE LA SECCIÓN XX}
%\label{SectionX_Y}
% For referencing this section elsewhere, use \ref{SectionX_Y}
% where X is the number of the chapter,
% and Y is the number of the section.

%\lhead[\thesection. NOMBRE DE LA SECCIÓN 1]{}

%Aquí se comienza a escribir el texto de la sección en cuestión.


% SUBSECTION XX
% -------------------------
%\subsection{Nombre del subtitulo XX}

%Aquí se comienza a escribir el texto de la subsección en cuestión.